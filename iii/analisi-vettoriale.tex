\chapter{Analisi vettoriale}
In questo capitolo studieremo come e quando è possibile estendere il teorema fondamentale del calcolo integrale, visto in uno dei precedenti capitoli, in più dimensioni, ovvero quando possiamo passare dall'integrazione di un insieme all'integrazione sulla sua frontiera.
Vedremo i principali risultati che si applicano all'analisi vettoriale in $\R^2$ e $\R^3$, in particolare i teoremi della divergenza (Gauss, Ostrogradskij) e del rotore (Green, Stokes), tutti casi particolari di un teorema di carattere generale noto come \emph{teorema di Stokes}, che riguarda anche le forme differenziali.

