\begin{figure}
	\tikzsetnextfilename{dini-esempio}
	\centering
	% Il codice di gnuplot per generare il grafico della funzione implicita è tratto da
	% gnuplot-surprising.blogspot.it/2011/09/assume-three-is-equation-fxygxy-and-we.html
	% (grazie cubanpit per averlo portato alla luce!)
	\begin{tikzpicture}
		\selectcolormodel{gray}
		% Non sapendo come cambiare il colore della linea disegnata da gnuplot,
		% questo comando forza il bianco/nero su tutto il grafico
		\begin{axis}[
				standard,
				xtick={0.6,1,2},
				xticklabels={$\bar{x}$,1,2},
				ytick={-2,-1,0.927},
				yticklabels={-2,-1,$\bar{y}$},
				xmin=-.2, xmax=2,
				ymin=-2, ymax=1.1, % Aggiungo .1 al massimo altrimenti la linea
									% appare tagliata a metà
				xlabel=$x$,
				ylabel=$y$
			]
			\addplot+[
				no markers,
				raw gnuplot,
				empty line=jump
			] gnuplot {
				set contour;
				set cntrparam levels discrete 0;
				set view map;
				unset surface;
				unset key;
				set isosamples 1000,1000;
				set xrange [0:2];
				set yrange [-2:1];
				splot cos(y)-x**3*y+x-1 with points lc variable;
			};
			% Le coordinate del punto "di massimo" a circa y=1 sono ~(0.6, 0.927)
			% (trovate risolvendo le equazioni con WolframAlpha)
			\draw[densely dotted] (.6,0) -- (.6,.927) -- (0,.927);
		\end{axis}
	\end{tikzpicture}
	\caption{Grafico (in parte) del luogo degli zeri di $F(x,y)=\cos y-x^3y+x-1$.
		Possiamo definire una funzione implicita $y=g(x)$ per ogni intorno di $(x,y)\ne(0,0)$, dato che nell'origine $\drp{F}{y}=0$; allo stesso modo però possiamo ottenere una funzione implicita $x=h(y)$ per $(x,y)\ne(\bar{x},\bar{y})$, poich\'e in tale punto si ha $\drp{F}{x}=0$.
		Nel secondo caso, ad esempio, in un intorno (qualunque) del punto è impossibile definire una funzione, dato che ad ogni $y$ corrisponderebbero due valori di $x$, uno a sinistra e uno a destra della retta $x=\bar{x}$.}
	\label{fig:dini-esempio}
\end{figure}

