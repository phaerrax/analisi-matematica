\begin{figure}
	\tikzsetnextfilename{taylor-sin}
	\centering
	\begin{tikzpicture}
		\begin{axis}[
			enlargelimits,axis on top,axis lines=middle,
			legend pos=outer north east,
			xlabel=$x$,xmin=-7,xmax=7,ymin=-2,ymax=2,
			ytick=\empty,xtick=\empty
		]
			\addplot [samples=500,color=black!15!white,domain=-2*pi:2*pi] function {x};
			\addplot [samples=500,color=black!30!white,domain=-2*pi:2*pi] function {x-(x^3)/6};
			\addplot [samples=500,color=black!45!white,domain=-2*pi:2*pi] function {x-(x^3)/6+(x^5)/120};
			\addplot [samples=500,color=black!60!white,domain=-2*pi:2*pi] function {x-(x^3)/6+(x^5)/120-(x^7)/5040};
			\addplot [samples=500,color=black!75!white,domain=-2*pi:2*pi] function {x-(x^3)/6+(x^5)/120-(x^7)/5040+(x^9)/362880};
			\addplot [samples=500,color=black!90!white,domain=-2*pi:2*pi] function {x-(x^3)/6+(x^5)/120-(x^7)/5040+(x^9)/362880-(x^11)/39916800};
			\addplot [very thick,samples=500,color=black,domain=-2*pi:2*pi] function {sin(x)};
			\legend{$n=0$,$n=1$,$n=2$,$n=3$,$n=4$,$n=5$,$\sin x$}
		\end{axis}
	\end{tikzpicture}
	\caption{Approssimazioni crescenti dello sviluppo di Taylor per $\sin x$. Si nota che lo sviluppo arrestato a $n=4$, che corrisponde alla nona potenza di $x$, già è un'ottima approssimazione in $(-\pi,\pi)$.}
	\label{fig:taylor_sin}
\end{figure}
