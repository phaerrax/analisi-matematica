\chapter{L'equazione del calore}
\label{ch:calore}
\newcommand{\sfc}{\Phi} % Soluzione fondamentale dell'equazione del calore

L'equazione del calore è
\begin{equation}
    \drp{u}{t}-\lap u=f
    \label{eq:calore}
\end{equation}
dove $f$ e $u$ sono funzioni di $(t,\vec x)$ da $[0,+\infty)\times\Omega$ a $\R$, con $\Omega\subset\R^n$.
La funzione
\begin{equation}
    \sfc(t,\vec x)=
    \begin{cases}
        0                                                                   & t\le 0\\
        \bigl(\frac1{4\pi t}\bigr)^\frac{n}2e^{-\frac{\norm{\vec x}}{4t}} & t>0
    \end{cases}
    \label{eq:soluzione-fondamentale-calore}
\end{equation}
risolve l'equazione \eqref{eq:calore} omogenea, con $f=0$: infatti
\begin{equation}
    \drp{\sfc}{t}(t,\vec x)=
    -\frac{n}{2t}\sfc(t,\vec x)+\frac{\norm{\vec x}^2}{4t^2}\sfc(t,\vec x)
\end{equation}
mentre
\begin{equation}
    \drp{\sfc}{x_i}(t,\vec x)=
    -\frac{x_i}{2t}\sfc(t,\vec x)
    \qqq
    \ddrp{\sfc}{x_i}(t,\vec x)=
    -\frac1{2t}\sfc(t,\vec x)+\frac{x_i^2}{4t^2}\sfc(t,\vec x)
\end{equation}
perciò
\begin{equation}
    \drp{\sfc}{t}(t,\vec x)-\lap\sfc(t,\vec x)=
    \biggl[-\frac{n}{2t}\sfc(t,\vec x)+\frac{\norm{\vec x}^2}{4t^2}-\sum_{i=1}^n\biggl(-\frac1{2t}\sfc(t,\vec x)+\frac{x_i^2}{4t^2}\biggr)\biggr]\sfc(t,\vec x)=
    0.
\end{equation}
La $\sfc$ è detta \emph{soluzione fondamentale dell'equazione del calore}: essa è radiale (in $\vec x$), ha una singolarità nell'origine, e
\begin{multline}
    \int_{\R^n}\sfc(t,\vec x)\,\dd^n x=
    \frac1{(4\pi t)^\frac{n}2}\int_{\R^n}e^{-\frac{\norm{\vec x}^2}{4t}}\,\dd^n x=
    \frac1{(4\pi t)^\frac{n}2}\int_{\R^n}4t^\frac{n}2e^{-\norm{\vzeta}^2}\,\dd^n\zeta=\\=
    \frac1{\pi^\frac{n}2}\int_{\R^n}e^{-\sum_{i=1}^n\zeta_i^2}\,\dd^n\zeta=
    \frac1{\pi^\frac{n}2}\biggl(\int_\R e^{-\zeta^2}\,\dd\zeta\biggr)^n=
    \frac1{\pi^\frac{n}2}(\sqrt{\pi})^n=
    1.
\end{multline}
Possiamo effettuare un riscalamento delle soluzioni: se $u=u(t,\vec x)$ risolve la \eqref{eq:calore}, allora anche $u_\lambda=u(\lambda^2 t,\lambda\vec x)$ la risolve, per ogni $\lambda\ne 0$.
Scegliendo $\lambda=\frac1{\sqrt{t}}$ allora si ha $u=u(1,\frac{\vec x}{t})$.
Consideriamo
\begin{equation}
    u(t,\vec x)=
    \lambda^{2\alpha}u(\lambda^2 t,\lambda\vec x)=
    t^{-\alpha}u\Bigl(1,\frac{\vec x}{t}\Bigr):
\end{equation}
sappiamo che $u$ dipende da $\vec x$ solo in modo radiale, perciò detto $r=\norm{\vec x}$ la \eqref{eq:calore} si riscrive come
\begin{equation}
    \drp{u}{t}-\ddrp{u}{r}-\frac{n-1}{r}\drp{u}{r}=0.
    \label{eq:calore-radiale}
\end{equation}
Sia $\vec y=\frac{\vec x}{t}$ e chiamiamo $v(\vec y)\defeq u(1,\vec y)$: allora
\begin{equation}
    -\alpha t^{-\alpha-1}v+t^{-\alpha}\biggl(-\frac{r}2t^{-\frac{3}{2}}\biggr)\drp{v}{y}-\frac{t^\alpha}{t}\ddrp{v}{y}-\frac{n-1}{r}\drp{v}{y}\frac{t^{-\alpha}}{\sqrt{t}}=0
\end{equation}
che si semplifica in
\begin{equation}
    \alpha v+\frac12y\drp{v}{y}+\ddrp{v}{y}+\frac{n-1}{y}\drp{v}{y}=0.
\end{equation}
Scegliendo $\alpha=\frac{n}2$ otteniamo
\begin{equation}
    \drp{}{y}\biggl(y^{n-1}\drp{v}{y}+\frac12y^nv\biggr)=0:
\end{equation}
se assumiamo che $v$ e la sua derivata, per $y\to+\infty$, tendano a zero più velocemente di un polinomio, allora la quantità derivata in quest'ultima equazione è nulla, da cui risulta
\begin{equation}
    \drp{v}{y}=-\frac12yv
\end{equation}
che ha come soluzione
\begin{equation}
    v(\vec y)=be^{-\frac14y^2}
\end{equation}
per un $b\in\R$ arbitrario, quindi, tornando alla forma iniziale della soluzione,
\begin{equation}
    u(t,\vec x)=\frac{b}{t^\frac{n}2}e^{-\frac{\norm{\vec x}^2}{4t}}
\end{equation}
che è la soluzione fondamentale, una volta trovato $b$ imponendo la normalizzazione per $\vec x\in\R^n$.

Prendiamo ora come condizione al contorno $u(0,\vec x)=g(\vec x)$: mostriamo ora che la funzione
\begin{equation}
    u(t,\vec x)=\int_{\R^n}\sfc(t,\vec x-\vec y)g(\vec y)\,\dd^ny
    \label{eq:soluzione-calore-convoluzione}
\end{equation}
risolve il problema di Cauchy.
Notiamo innanzitutto che in senso distribuzionale si ha
\begin{equation}
    \lim_{(t,\vec x)\to(0,\vec 0)}\sfc(t,\vec x)=\delta(\vec x).
\end{equation}
Sia $g\in\cont[\infty]{\R^n}\cap\leb[\infty]{\R^n}$, ossia liscia e limitata.
Al di fuori della singolarità, ossia per $t\ne 0$, si avrà anche che $u\in\cont[\infty]{(0,+\infty)\times\R^n}$, e per il limite precedente $\lim_{(t,\vec x)\to(0,\vec x_0)}u(t,\vec x)=g(\vec x_0)$.
Fissiamo dunque $\delta>0$ e prendiamo $t\in[\delta,+\infty)$: chiaramente la funzione $\sfc$ è $\cclass[\infty]$ su $[\delta,+\infty)\times\R^n$, per ogni $\delta>0$, perciò lo è anche su tutto $(0,+\infty)\times\R^n$, e si ha
\begin{equation}
    \biggl(\drp{}{t}-\lap\biggr)u(t,\vec x)=
    \int_{\R^n}\biggl(\drp{}{t}-\lap\biggr)\sfc(t,\vec x-\vec y)g(\vec y)\,\dd^ny=
    0
\end{equation}
dato che $\sfc$ è soluzione della \eqref{eq:calore}.
Fissiamo poi $\vec x_0\in\R^n$ ed $\epsilon>0$: per la continuità di $g$, esiste $\delta>0$ tale che $\abs{g(\vec y)-g(\vec x_0)}<\epsilon$ se $\norm{\vec y-\vec x_0}<\delta$; preso dunque $\vec x$ tale che $\norm{\vec x-\vec x_0}<\frac{\delta}2$ possiamo allora stimare
\begin{equation}
    \begin{split}
        \abs{u(t,\vec x)-g(\vec x_0)}&=
        \abs[\bigg]{\int_{\R^n}\sfc(t,\vec x-\vec y)g(\vec y)\,\dd^ny-g(\vec x_0)\int_{\R^n}\sfc(\vec t,\vec x-\vec y)\,\dd^ny}=\\ &=
        \abs[\bigg]{\int_{\R^n}\sfc(t,\vec x-\vec y)\bigl[g(\vec y)-g(\vec x_0)\bigr]\,\dd^ny}\le\\ &\le
        \int_{B_\delta(\vec x_0)}\sfc(t,\vec x-\vec y)\abs{g(\vec y)-g(\vec x_0)}\,\dd^ny+
        \int_{\R^n\setminus B_\delta(\vec x_0)}\sfc(t,\vec x-\vec y)\abs{g(\vec y)-g(\vec x_0)}\,\dd^ny.
    \end{split}
\end{equation}
Nel primo integrale abbiamo $\vec y\in B_\delta(\vec x_0)$ dunque $\abs{g(\vec y)-g(\vec x_0)}<\epsilon$: siccome $\sfc>0$ su tutto $\R^n$ allora
\begin{equation}
    \int_{B_\delta(\vec x_0)}\sfc(t,\vec x-\vec y)\abs{g(\vec y)-g(\vec x_0)}\,\dd^ny< 
    \epsilon\int_{B_\delta(\vec x_0)}\sfc(t,\vec x-\vec y)\,\dd^ny\le
    \epsilon\int_{\R^n}\sfc(t,\vec x-\vec y)\,\dd^ny=
    \epsilon.
\end{equation}
Nel secondo, siccome $g$ è limitata si ha per la disuguaglianza triangolare che $\abs{g(\vec y)-g(\vec x_0)}\le\abs{g(\vec y)}+\abs{g(\vec x_0)}\le 2\norm{g}_\infty$.
Abbiamo inoltre $\norm{\vec x-\vec x_0}<\frac{\delta}2$ e $\norm{\vec x-\vec x_0}>\frac{\delta}2$ quindi
\begin{equation}
    \norm{\vec y-\vec x_0}<
    \norm{\vec y-\vec x}+\norm{\vec x-\vec x_0}<
    \norm{\vec y-\vec x}+\frac{\delta}2<
    \norm{\vec y-\vec x}+\frac{\norm{\vec y-\vec x_0}}2
    \qqq
    \frac12\norm{\vec y-\vec x_0}<\norm{\vec y-\vec x}.
\end{equation}
Raggruppando tutte le costanti in una $c\in\R$ risulta allora che il secondo integrale non è maggiore di
\begin{equation}
    \frac{c}{t^\frac{n}2}\int_{\R^n\setminus B_\delta(\vec x_0)}e^{-\frac{\norm{\vec x-\vec y}^2}{4t}}\,\dd^ny\le
    \frac{c}{t^\frac{n}2}\int_{\R^n\setminus B_\delta(\vec x_0)}e^{-\frac{\norm{\vec y-\vec x_0}^2}{16t}}\,\dd^ny
\end{equation}
che è indipendente da $\vec x$, inoltre è radiale con centro in $\vec x_0$: detta $r\defeq\norm{\vec y-\vec x_0}$ quest'ultimo integrale è uguale a
\footnote{
    Per ogni $t>0$ e $p\in\N_0$ si ha la stima $e^t\ge\frac1{p!}t^p$: si osservi semplicemente che il secondo membro è uno dei termini (tutti positivi per $t>0$) dello sviluppo di Taylor di $e^t$.
}
\begin{multline}
    \frac{c}{t^\frac{n}2}\int_\delta^{+\infty}e^{-\frac{r^2}{16t}}r^{n-1}\,\dd r\le
    \frac{c}{t^\frac{n}2}\int_\delta^{+\infty}\frac1{\frac1{n!}\bigl(\frac{r^2}{16t}\bigr)^n}r^{n-1}\,\dd r=\\=
    \frac{cn!16^n}{t^{\frac{n}2-n}}\int_\delta^{+\infty}\frac1{r^{n+1}}\,\dd r=
    \tilde{c}t^\frac{n}2\frac1{r^n}\bigg|_\delta^{+\infty}=
    \tilde{c}\delta^{-n}t^\frac{n}2
\end{multline}
che è arbitrariamente piccolo per $t\to 0$.
\begin{osservazione} \label{o:propagazione-calore-velocita-infinita}
    Se $g\ge 0$ in un intorno dell'origine e $g=0$ al di fuori, allora per ogni $(t,\vec x)\in(0,+\infty)\times\R^n$ si ha $u(t,\vec x)>0$ poich\'e $\sfc(t,\vec x-\vec y)>0$ e $g(\vec y)>0$: secondo questo modello dunque la velocità di propagazione del calore è infinita!
\end{osservazione}
