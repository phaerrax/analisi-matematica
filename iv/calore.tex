\chapter{L'equazione del calore}
\label{ch:calore}
\newcommand{\sfc}{\Phi} % Soluzione fondamentale dell'equazione del calore

L'equazione del calore è
\begin{equation}
    \drp{u}{t}-\lap u=f
    \label{eq:calore}
\end{equation}
dove $f$ e $u$ sono funzioni di $(t,\vec x)$ da $[0,+\infty)\times\Omega$ a $\R$, con $\Omega\subset\R^n$.
La funzione
\begin{equation}
    \sfc(t,\vec x)=
    \begin{cases}
        0                                                                   & t\le 0\\
        \bigl(\frac1{4\pi t}\bigr)^\frac{n}2e^{-\frac{\norm{\vec x}}{4t}} & t>0
    \end{cases}
    \label{eq:soluzione-fondamentale-calore}
\end{equation}
risolve l'equazione \eqref{eq:calore} omogenea, con $f=0$: infatti
\begin{equation}
    \drp{\sfc}{t}(t,\vec x)=
    -\frac{n}{2t}\sfc(t,\vec x)+\frac{\norm{\vec x}^2}{4t^2}\sfc(t,\vec x)
\end{equation}
mentre
\begin{equation}
    \drp{\sfc}{x_i}(t,\vec x)=
    -\frac{x_i}{2t}\sfc(t,\vec x)
    \qqq
    \ddrp{\sfc}{x_i}(t,\vec x)=
    -\frac1{2t}\sfc(t,\vec x)+\frac{x_i^2}{4t^2}\sfc(t,\vec x)
\end{equation}
perciò
\begin{equation}
    \drp{\sfc}{t}(t,\vec x)-\lap\sfc(t,\vec x)=
    \biggl[-\frac{n}{2t}\sfc(t,\vec x)+\frac{\norm{\vec x}^2}{4t^2}-\sum_{i=1}^n\biggl(-\frac1{2t}\sfc(t,\vec x)+\frac{x_i^2}{4t^2}\biggr)\biggr]\sfc(t,\vec x)=
    0.
\end{equation}
La $\sfc$ è detta \emph{soluzione fondamentale dell'equazione del calore}: essa è radiale (in $\vec x$), ha una singolarità nell'origine, e
\begin{multline}
    \int_{\R^n}\sfc(t,\vec x)\,\dd^n x=
    \frac1{(4\pi t)^\frac{n}2}\int_{\R^n}e^{-\frac{\norm{\vec x}^2}{4t}}\,\dd^n x=
    \frac1{(4\pi t)^\frac{n}2}\int_{\R^n}4t^\frac{n}2e^{-\norm{\vzeta}^2}\,\dd^n\zeta=\\=
    \frac1{\pi^\frac{n}2}\int_{\R^n}e^{-\sum_{i=1}^n\zeta_i^2}\,\dd^n\zeta=
    \frac1{\pi^\frac{n}2}\biggl(\int_\R e^{-\zeta^2}\,\dd\zeta\biggr)^n=
    \frac1{\pi^\frac{n}2}(\sqrt{\pi})^n=
    1.
\end{multline}
Possiamo effettuare un riscalamento delle soluzioni: se $u=u(t,\vec x)$ risolve la \eqref{eq:calore}, allora anche $u_\lambda=u(\lambda^2 t,\lambda\vec x)$ la risolve, per ogni $\lambda\ne 0$.
Scegliendo $\lambda=\frac1{\sqrt{t}}$ allora si ha $u=u(1,\frac{\vec x}{t})$.
Consideriamo
\begin{equation}
    u(t,\vec x)=
    \lambda^{2\alpha}u(\lambda^2 t,\lambda\vec x)=
    t^{-\alpha}u\Bigl(1,\frac{\vec x}{t}\Bigr):
\end{equation}
sappiamo che $u$ dipende da $\vec x$ solo in modo radiale, perciò detto $r=\norm{\vec x}$ la \eqref{eq:calore} si riscrive come
\begin{equation}
    \drp{u}{t}-\ddrp{u}{r}-\frac{n-1}{r}\drp{u}{r}=0.
    \label{eq:calore-radiale}
\end{equation}
Sia $\vec y=\frac{\vec x}{t}$ e chiamiamo $v(\vec y)\defeq u(1,\vec y)$: allora
\begin{equation}
    -\alpha t^{-\alpha-1}v+t^{-\alpha}\biggl(-\frac{r}2t^{-\frac{3}{2}}\biggr)\drp{v}{y}-\frac{t^\alpha}{t}\ddrp{v}{y}-\frac{n-1}{r}\drp{v}{y}\frac{t^{-\alpha}}{\sqrt{t}}=0
\end{equation}
che si semplifica in
\begin{equation}
    \alpha v+\frac12y\drp{v}{y}+\ddrp{v}{y}+\frac{n-1}{y}\drp{v}{y}=0.
\end{equation}
Scegliendo $\alpha=\frac{n}2$ otteniamo
\begin{equation}
    \drp{}{y}\biggl(y^{n-1}\drp{v}{y}+\frac12y^nv\biggr)=0:
\end{equation}
se assumiamo che $v$ e la sua derivata, per $y\to+\infty$, tendano a zero più velocemente di un polinomio, allora la quantità derivata in quest'ultima equazione è nulla, da cui risulta
\begin{equation}
    \drp{v}{y}=-\frac12yv
\end{equation}
che ha come soluzione
\begin{equation}
    v(\vec y)=be^{-\frac14y^2}
\end{equation}
per un $b\in\R$ arbitrario, quindi, tornando alla forma iniziale della soluzione,
\begin{equation}
    u(t,\vec x)=\frac{b}{t^\frac{n}2}e^{-\frac{\norm{\vec x}^2}{4t}}
\end{equation}
che è la soluzione fondamentale, una volta trovato $b$ imponendo la normalizzazione per $\vec x\in\R^n$.

Prendiamo ora come condizione al contorno $u(0,\vec x)=g(\vec x)$: mostriamo ora che la funzione
\begin{equation}
    u(t,\vec x)=\int_{\R^n}\sfc(t,\vec x-\vec y)g(\vec y)\,\dd^ny
    \label{eq:soluzione-calore-convoluzione}
\end{equation}
risolve il problema di Cauchy.
Notiamo innanzitutto che in senso distribuzionale si ha
\begin{equation}
    \lim_{(t,\vec x)\to(0,\vec 0)}\sfc(t,\vec x)=\delta(\vec x).
\end{equation}
Sia $g\in\cont[\infty]{\R^n}\cap\leb[\infty]{\R^n}$, ossia liscia e limitata.
Al di fuori della singolarità, ossia per $t\ne 0$, si avrà anche che $u\in\cont[\infty]{(0,+\infty)\times\R^n}$, e per il limite precedente $\lim_{(t,\vec x)\to(0,\vec x_0)}u(t,\vec x)=g(\vec x_0)$.
Fissiamo dunque $\delta>0$ e prendiamo $t\in[\delta,+\infty)$: chiaramente la funzione $\sfc$ è $\cclass[\infty]$ su $[\delta,+\infty)\times\R^n$, per ogni $\delta>0$, perciò lo è anche su tutto $(0,+\infty)\times\R^n$, e si ha
\begin{equation}
    \biggl(\drp{}{t}-\lap\biggr)u(t,\vec x)=
    \int_{\R^n}\biggl(\drp{}{t}-\lap\biggr)\sfc(t,\vec x-\vec y)g(\vec y)\,\dd^ny=
    0
\end{equation}
dato che $\sfc$ è soluzione della \eqref{eq:calore}.
Fissiamo poi $\vec x_0\in\R^n$ ed $\epsilon>0$: per la continuità di $g$, esiste $\delta>0$ tale che $\abs{g(\vec y)-g(\vec x_0)}<\epsilon$ se $\norm{\vec y-\vec x_0}<\delta$; preso dunque $\vec x$ tale che $\norm{\vec x-\vec x_0}<\frac{\delta}2$ possiamo allora stimare
\begin{equation}
    \begin{split}
        \abs{u(t,\vec x)-g(\vec x_0)}&=
        \abs[\bigg]{\int_{\R^n}\sfc(t,\vec x-\vec y)g(\vec y)\,\dd^ny-g(\vec x_0)\int_{\R^n}\sfc(\vec t,\vec x-\vec y)\,\dd^ny}=\\ &=
        \abs[\bigg]{\int_{\R^n}\sfc(t,\vec x-\vec y)\bigl[g(\vec y)-g(\vec x_0)\bigr]\,\dd^ny}\le\\ &\le
        \int_{B_\delta(\vec x_0)}\sfc(t,\vec x-\vec y)\abs{g(\vec y)-g(\vec x_0)}\,\dd^ny+
        \int_{\R^n\setminus B_\delta(\vec x_0)}\sfc(t,\vec x-\vec y)\abs{g(\vec y)-g(\vec x_0)}\,\dd^ny.
    \end{split}
\end{equation}
Nel primo integrale abbiamo $\vec y\in B_\delta(\vec x_0)$ dunque $\abs{g(\vec y)-g(\vec x_0)}<\epsilon$: siccome $\sfc>0$ su tutto $\R^n$ allora
\begin{equation}
    \int_{B_\delta(\vec x_0)}\sfc(t,\vec x-\vec y)\abs{g(\vec y)-g(\vec x_0)}\,\dd^ny< 
    \epsilon\int_{B_\delta(\vec x_0)}\sfc(t,\vec x-\vec y)\,\dd^ny\le
    \epsilon\int_{\R^n}\sfc(t,\vec x-\vec y)\,\dd^ny=
    \epsilon.
\end{equation}
Nel secondo, siccome $g$ è limitata si ha per la disuguaglianza triangolare che $\abs{g(\vec y)-g(\vec x_0)}\le\abs{g(\vec y)}+\abs{g(\vec x_0)}\le 2\norm{g}_\infty$.
Abbiamo inoltre $\norm{\vec x-\vec x_0}<\frac{\delta}2$ e $\norm{\vec x-\vec x_0}>\frac{\delta}2$ quindi
\begin{equation}
    \norm{\vec y-\vec x_0}<
    \norm{\vec y-\vec x}+\norm{\vec x-\vec x_0}<
    \norm{\vec y-\vec x}+\frac{\delta}2<
    \norm{\vec y-\vec x}+\frac{\norm{\vec y-\vec x_0}}2
    \qqq
    \frac12\norm{\vec y-\vec x_0}<\norm{\vec y-\vec x}.
\end{equation}
Raggruppando tutte le costanti in una $c\in\R$ risulta allora che il secondo integrale non è maggiore di
\begin{equation}
    \frac{c}{t^\frac{n}2}\int_{\R^n\setminus B_\delta(\vec x_0)}e^{-\frac{\norm{\vec x-\vec y}^2}{4t}}\,\dd^ny\le
    \frac{c}{t^\frac{n}2}\int_{\R^n\setminus B_\delta(\vec x_0)}e^{-\frac{\norm{\vec y-\vec x_0}^2}{16t}}\,\dd^ny
\end{equation}
che è indipendente da $\vec x$, inoltre è radiale con centro in $\vec x_0$: detta $r\defeq\norm{\vec y-\vec x_0}$ quest'ultimo integrale è uguale a
\footnote{
    Per ogni $t>0$ e $p\in\N_0$ si ha la stima $e^t\ge\frac1{p!}t^p$: si osservi semplicemente che il secondo membro è uno dei termini (tutti positivi per $t>0$) dello sviluppo di Taylor di $e^t$.
}
\begin{multline}
    \frac{c}{t^\frac{n}2}\int_\delta^{+\infty}e^{-\frac{r^2}{16t}}r^{n-1}\,\dd r\le
    \frac{c}{t^\frac{n}2}\int_\delta^{+\infty}\frac1{\frac1{n!}\bigl(\frac{r^2}{16t}\bigr)^n}r^{n-1}\,\dd r=\\=
    \frac{cn!16^n}{t^{\frac{n}2-n}}\int_\delta^{+\infty}\frac1{r^{n+1}}\,\dd r=
    \tilde{c}t^\frac{n}2\frac1{r^n}\bigg|_\delta^{+\infty}=
    \tilde{c}\delta^{-n}t^\frac{n}2
\end{multline}
che è arbitrariamente piccolo per $t\to 0$.
\begin{osservazione} \label{o:propagazione-calore-velocita-infinita}
    Se $g\ge 0$ in un intorno dell'origine e $g=0$ al di fuori, allora per ogni $(t,\vec x)\in(0,+\infty)\times\R^n$ si ha $u(t,\vec x)>0$ poich\'e $\sfc(t,\vec x-\vec y)>0$ e $g(\vec y)>0$: secondo questo modello dunque la velocità di propagazione del calore è infinita!
\end{osservazione}

Affrontiamo ora l'equazione non omogenea \eqref{eq:calore} con $f\ne 0$, e $u(0,\vec x)=0$ $\forall\vec x\in\R^n$.
Seguendo il principio di Duhamel, usiamo la $f$ come dato iniziale per ogni istante di tempo, ossia prendiamo la funzione
\begin{equation}
    u(t,\vec x)=\int_0^t\int_{\R^n}\sfc(t-s,\vec x-\vec y)f(s,\vec y)\,\dd^ny\,\dd s.
\end{equation}
Assumiamo che la mappa $f\mapsto f(t,\vec x)$ sia $\cclass[2]$ in $[0,+\infty)$, che la mappa $\vec x\mapsto f(t,\vec x)$ sia $\cclass[1]$ in $\R^n$, e che $\supp f$ sia compatto in $[0,+\infty)\times\R^n$.
Con il cambio di variabile $z\defeq\vec x-\vec y$ e $r\defeq t-s$ abbiamo
\begin{equation}
    u(t,\vec x)=
    \int_0^t\int_{\R^n}\sfc(r,\vec z)f(t-r,\vec x-\vec z)\,\dd^nz\,\dd r
\end{equation}
e siccome il supporto di $f$ è compatto nell'insieme di integrazione si ha
\begin{equation}
    \drp{u}{t}(t,\vec x)=\int_0^t\int_{\R^n}\sfc(r,\vec z)\drp{f}{t}(t-r,\vec x-\vec z)\,\dd^nz\,\dd r+\int_{\R^n}\sfc(t,\vec z)f(0,\vec x-\vec z)\,\dd^nz.
\end{equation}
In modo simile si calcola
\begin{equation}
    \frac{\partial^2 u}{\partial x_i\partial x_j}=
    \int_0^t\int_{\R^n}\sfc(r,\vec z)\frac{\partial^2 f}{\partial x_i\partial x_j}(t-r,\vec x-\vec z)\,\dd^nz\,\dd r.
\end{equation}
Data la regolarità delle funzioni integrande, risulta che $u$ è $\cclass[1]$ nella variabile $t\in(0,+\infty)$ e $\cclass[2]$ nella variabile $\vec x\in\R^n$.
Sostituendo quanto trovato nella \eqref{eq:calore} si ha allora
\begin{equation}
    \begin{split}
        \biggl(\drp{}{t}-\lap\biggr)u(t,\vec x)&=
        \int_0^t\int_{\R^n}\sfc(r,\vec z)\biggl(\drp{}{t}-\lap\biggr)f(t-r,\vec x-\vec x)+\int_{\R^n}\sfc(t,\vec z)f(0,\vec x-\vec z)\,\dd^nz=\\ &=
        \!\begin{multlined}[t]
            \int_\epsilon^t\int_{\R^n}\sfc(r,\vec z)\biggl(-\drp{}{r}-\lap\biggr)f(t-r,\vec x-\vec z)+\\+
            \int_0^\epsilon\int_{\R^n}\sfc(r,\vec z)\biggl(-\drp{}{r}-\lap\biggr)f(t-r,\vec x-\vec x)+\int_{\R^n}\sfc(t,\vec z)f(0,\vec x-\vec z)\,\dd^nz
        \end{multlined}
    \end{split}
\end{equation}
per un $\epsilon>0$.
Analizziamo i tre termini separatamente: dal primo, integrando per parti (i termini di bordo dovuti al laplaciano sono nulli poich\'e $f$ ha supporto compatto in $\R^n$) otteniamo
\begin{equation}
    \begin{aligned}
        &\int_\epsilon^t\int_{\R^n}\sfc(r,\vec z)\biggl(\drp{}{t}-\lap_{\vec x}\biggr)f(t-r,\vec x-\vec x)=\\
        =&\int_\epsilon^t\int_{\R^n}\biggl(\drp{}{r}-\lap_{\vec z}\biggr)\sfc(r,\vec z)f(t-r,\vec x-\vec z)\,\dd^nz\,\dd r-\int_{\R^n}\sfc(r,\vec z)f(t-r,\vec x-\vec z)\,\dd^nz\bigg|_\epsilon^t=\\
        =&-\int_{\R^n}\sfc(t,\vec z)f(0,\vec x-\vec z)\,\dd^nz+\int_{\R^n}\sfc(\epsilon,\vec z)f(t-\epsilon,\vec x-\vec z)\,\dd^nz,
    \end{aligned}
\end{equation}
e il primo di questi due termini si cancella con l'ultimo dei tre integrali sopra.
Per il secondo termine, il suo valore assoluto non è maggiore di
\begin{equation}
    \begin{aligned}
        &\int_0^\epsilon\int_{\R^n}\sfc(r,\vec z)\biggl(\drp{}{t}-\lap_{\vec x}\biggr)f(t-r,\vec x-\vec x)\le\\
        \le&\int_0^\epsilon\int_{\R^n}\sfc(r,\vec z)\biggl[\abs[\bigg]{\drp{f}{r}(t-r,\vec x-\vec z)}+\abs[\big]{\lap_{\vec x}f(t-r,\vec x\vec z)}\biggr]\,\dd^nz\,\dd r\le\\
        \le&\biggl(\norm[\bigg]{\drp{f}{r}}_\infty+\norm{\lap f}_\infty\biggr)\int_0^\epsilon\int_{\R^n}\sfc(r,\vec z)\,\dd^nz\,\dd r=\\
        =&\biggl(\norm[\bigg]{\drp{f}{r}}_\infty+\norm{\lap f}_\infty\biggr)\epsilon.
    \end{aligned}
\end{equation}
In conclusione si ha, come in precedenza,
\begin{equation}
    \biggl(\drp{}{t}-\lap\biggr)u(t,\vec x)=
    \lim_{\epsilon\to 0}\int_{\R^n}\sfc(\epsilon,\vec z)f(t-\epsilon,\vec x-\vec z)\,\dd^nz=
    f(t,\vec z)
\end{equation}
e per $t\to 0$
\begin{equation}
    \abs{u(t,\vec x)}\le
    \norm{u}_\infty\le
    t\norm{f}_\infty\int_{\R^n}\sfc(s,\vec y)\,\dd^ny\to
    0.
\end{equation}
Per la linearità dell'equazione possiamo comporre le due soluzioni trovate nei due casi: con tutte le ipotesi fatte, la funzione
\begin{equation}
    u(t,\vec x)=\int_{\R^n}\sfc(t,\vec x-\vec y)g(\vec y)\,\dd^ny+\int_0^t\int_{\R^n}\sfc(t-s,\vec x-\vec y)f(s,\vec y)\,\dd^ny\,\dd s
    \label{eq:soluzione-generale-equazione-calore}
\end{equation}
risolve il problema di Cauchy
\begin{equation}
    \begin{cases}
        \drp{u}{t}(t,\vec x)-\lap u(t,\vec x)=f(t,\vec x) & (t,\vec x)\in(0,+\infty)\times\R^n\\
        u(0,\vec x)=g(\vec x)                             & \vec x\in\R^n
    \end{cases}.
    \label{eq:problema-cauchy-equazione-calore}
\end{equation}

\section{Equazioni paraboliche}
L'equazione del calore è un'equazione parabolica del secondo ordine.
Prendiamo $\Omega\subset\R^n$ limitato, e chiamiamo $\Omega_T\defeq\Omega\times(0,T)$ per $T>0$.
Consideriamo l'operatore differenziale
\begin{equation}
    L\colon u\mapsto-\sum_{i=1}^n\sum_{j=1}^n\drp{}{x_i}\biggl(a_{ij}\drp{u}{x_j}\biggr)+\sum_{i=1}^nb_i\drp{u}{x_i}+cu
\end{equation}
con $a_{ij}=a_{ji}$ per ogni $\vec x\in\Omega$ e con $\norm{a_{ij}}_\infty,\norm{b_i}_\infty,\norm{c}_\infty<+\infty$.
Un'equazione della forma
\begin{equation}
    \drp{u}{t}+L(u)=f
\end{equation}
è detta parabolica se $L$ è ellittico.
In questa definizione i coefficienti di $L$ possono benissimo anche dipendere da $t$, ma la variabile non entra mai nelle derivate.
Il problema di Cauchy si scrive come
\begin{equation}
    \begin{cases}
        \drp{u}{t}+L(u)=f & \text{in }\Omega_T\\
        u=0               & \text{in }\boundary\Omega\times(0,T]\\
        u=g               & \text{in }\Omega\times\{0\}
    \end{cases}
    \label{eq:problema-cauchy-parabolico}
\end{equation}
per $f\in\leb[2]{\Omega_T}$, $g\in\leb[2]{\Omega}$.
Definiamo inoltre la forma bilineare $B:\sobHc[1]{\Omega}\times\sobHc[1]{\Omega}\times[0,T]\to\R$ come
\begin{equation}
    B(u,v,t)\defeq\int_\Omega\biggl(\sum_{i=1}^n\sum_{j=1}^na_{ij}\drp{u}{x_i}\drp{v}{x_j}+\sum_{i=1}^nb_i\drp{u}{x_i}v+cuv\biggr)\,\dd\mu
    \label{eq:forma-bilineare-parabolica}
\end{equation}
che è continua, e dipende da $t$ ma non da $\vec x$.
Scriviamo l'equazione parabolica in forma debole come
\begin{equation}
    \inner[\bigg]{\drp{u}{t}}{v}+B(u,v,t)=\inner{f}{v}
    \label{eq:equazione-parabolica-debole}
\end{equation}
e $u$ è una soluzione debole del problema se vale questa equazione (insieme alla condizione al contorno $u(0,\vec x)=g(\vec x)$ $\forall\vec x\in\R^n$) per quasi ogni $t\in[0,T]$ e per ogni $v\in\sobHc[1]{\Omega}$.
Possiamo considerare il prodotto interno come quello in $\lclass[2]$, ma possiamo anche considerare $\inner{f}{v}$ e $\inner[\big]{\drp{u}{t}}{v}$ come l'azione di funzionali su $v$, e dato che $\sobHc[1]{\Omega}\subset\sobH[1]{\Omega}$ la classe di funzionali è più ampia dello spazio $\sobHc[1]{\Omega}$ stesso: usando la notazione $\sobH[-1]{\Omega}\defeq\dual{\sobHc[1]{\Omega}}$, si ha infatti
\begin{equation}
    \sobHc[1]{\Omega}\subset\leb[2]{\Omega}\subset\sobH[-1]{\Omega}
\end{equation}
e cos\`i possiamo sostituire il prodotto interno $\inner{f}{v}$ con la \emph{forma di dualità} $\dualpair{f}{v}\defeq f(v)$.
Per distinguere poi la dipendenza da $t$ da quella da $\vec x$, che sono ben diverse, consideriamo $u$ come una funzione che associa a $t\in[0,T]$ una funzione $\tilde{u}(t)$ tale che
\begin{equation}
    [\tilde{u}(t)](\vec x)=u(t,\vec x)
\end{equation}
separando perciò la mappa $t\mapsto\tilde{u}(t)$ che è in $\leb[2]{0,T;\sobHc[1]{\Omega}}$, dalla mappa $\vec x\mapsto[\tilde{u}(t)](\vec x)=u(t,\vec x)$ che è in $\sobHc[1]{\Omega}$.
Per quanto appena detto sugli spazi duali, possiamo allora estendere la derivata $\drp{u}{t}$ allo stesso modo come una funzione in $\leb[2]{0,T;\sobH[-1]{\Omega}}$.
\begin{teorema}
    Siano $u\in\leb[2]{0,T;\sobHc[1]{\Omega}}$ e $\drp{u}{t}\in\leb[2]{0,T;\sobH[-1]{\Omega}}$.
    Allora $u\in\cont{0,T;\leb[2]{\Omega}}$ e la mappa $t\mapsto\norm{u(t,\cdot)}_2^2$ è assolutamente continua, vale a dire
    \begin{equation}
        \drv{}{t}\norm{u(t,\cdot)}_2^2=2\dualpair[\bigg]{\drp{u}{t}(t)}{u(t)}.
    \end{equation}
\end{teorema}
