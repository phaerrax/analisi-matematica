\chapter{L'equazione delle onde}
\label{ch:onde}

L'equazione delle onde, o anche di D'Alembert, è
\begin{equation}
    \ddrp{u}{t}-\lap u=f
    \label{eq:onde}
\end{equation}
per $u=u(t,\vec x)$ con $t\in[0,+\infty)$ e $\vec x\in\Omega\subset\R^n$.
Più in breve, definendo l'\emph{operatore dalembertiano} $\dal=\ddrp{}{t}-\lap$ possiamo riscriverla come
\begin{equation}
    \dal u=f.
    \label{eq:onde-dalembertiano}
\end{equation}
Per semplicità, trattiamo il caso $n=1$, $\Omega=\R$ e poniamo il problema di Cauchy
\begin{equation}
    \begin{cases}
        \ddrp{u}{t}-\ddrp{u}{x}=0 & \text{in }(0,+\infty)\times\R\\
        u=g                       & \text{in }\{0\}\times\R\\
        \drp{u}{t}=h              & \text{in }\{0\}\times\R
    \end{cases}
    \label{eq:problema-cauchy-onde}
\end{equation}
e cerchiamo una soluzione in termini di $g$ e $h$.
Mediante un'opportuna fattorizzazione possiamo ricondurre l'equazione omogenea a una coppia di equazioni del trasporto:
\begin{equation}
    \dal u=\biggl(\drp{}{t}+\drp{}{x}\biggr)\biggl(\drp{u}{t}-\drp{u}{x}\biggr)=0
\end{equation}
e detto $v=\drp{u}{t}-\drp{u}{x}$ abbiamo
\begin{equation}
    \drp{v}{t}+\drp{v}{x}=0.
\end{equation}
Riconducendoci all'esempio \ref{es:equazione-trasporto}, sia $a(x-t)\defeq v(t,x)$; risulta $\drp{v}{x}(t,x)=a'(x-t)$ e $\drp{v}{t}(t,x)=-a'(t,x)$.
D'altro canto l'equazione
\begin{equation}
    \drp{u}{t}(t,x)-\drp{u}{x}(t,x)=a(x-t)
\end{equation}
è ancora un'equazione del trasporto (non omogenea), per cui
\begin{equation}
    u(t,x)=
    \int_0^ta\bigl(x+(t-s)-s)\,\dd s+b(x+t)=
    \frac12\int_{x-t}^{x+t}a(y)\,\dd y+b(x+t)
\end{equation}
dove $a$ è determinata dalla non omogeneità della soluzione mentre $b$ è determinata dai dati iniziali: si ha $b(x)=u(0,x)=g(x)$ e $a(x)=v(0,x)=\drp{u}{t}(0,x)-\drp{u}{x}(0,x)=h(x)-g'(x)$ quindi
\begin{equation}
    \begin{split}
        u(t,x)&=
        \frac12\int_{x-t}^{x+t}\bigl[h(y)-g'(y)\bigr]\,\dd y+g(x+t)=\\ &=
        \frac12\int_{x-t}^{x+t}h(y)\,\dd y+\frac12\bigl[g(x-t)-g(x+t)\bigr]+g(x+t)=\\ &=
        \frac12\int_{x-t}^{x+t}h(y)\,\dd y+\frac12\bigl[g(x-t)+g(x+t)\bigr]
    \end{split}
    \label{eq:rappresentazione-soluzione-onde}
\end{equation}
che è la formula di rappresentazione generale.
Essa vale per $t>0$, ma si dimostra che se $g\in\cont[2]{\R}$ e $h\in\cont[1]{\R}$ allora $u$ soddisfa l'equazione delle onde su tutto $\R$.
\begin{teorema}
    Se $g\in\cclass[2]$ e $h\in\cclass[1]$ allora la soluzione $u$ del problema di Cauchy \eqref{eq:problema-cauchy-onde} è in $\cont[2]{(0,+\infty)\times\R}$ e vale
    \begin{equation}
        g(x_0)=\lim_{(t,x)\to(0,x_0)}u(t,x)
        \qtext{e}
        h(x_0)=\lim_{(t,x)\to(0,x_0)}\drp{u}{t}(t,x).
    \end{equation}
\end{teorema}

\section{Equazioni iperboliche}
Generalizziamo l'equazione delle onde a un generico problema
\begin{equation}
    \begin{cases}
        \ddrp{u}{t}+L(u)=f & \text{in }\Omega_T=[0,T]\times\Omega\\
        u=0                & \text{in }[0,T]\times\boundary\Omega\\
        u=g                & \text{in }\{0\}\times\Omega\\
        \drp{u}{t}=h       & \text{in }\{0\}\times\Omega
    \end{cases}
    \label{eq:problema-cauchy-iperbolico}
\end{equation}
con $f\colon\Omega_T\to\R$ e $g,h\colon\Omega\to\R$, ed $L$ un operatore ellittico dipendente da $t$:
\begin{equation}
    L\colon u\mapsto -\sum_{i=1}^n\sum_{j=1}^n\drp{}{x_i}\biggl(a_{ij}\drp{u}{x_j}\biggr)+\sum_{i=1}^nb_i\drp{u}{x_i}+cu
\end{equation}
tale che esiste $\lambda>0$ per cui
\begin{equation}
    \sum_{i=1}^n\sum_{j=1}^n a_{ij}(t,\vec x)\xi_i\xi_j \ge \lambda\norm{\vxi}^2
\end{equation}
per ogni $\vxi\in\R^n$ e $(t,\vec x)\in\Omega_T$.
Cerchiamone una soluzione debole, assumendo che $a_{ji}=a_{ij}$ e $a_{ij},b_i\in\cont[1]{\clos{\Omega_T}}$, $f\in\leb[2]{\Omega_T}$, $g\in\sobHc[1]{\Omega}$ e $h\in\leb[2]{\Omega}$.
Ricaviamo dall'equazione la forma bilineare
\begin{equation}
    B(u,v,t)=\int_\Omega\biggl[\sum_{i=1}^n\sum_{j=1}^na_{ij}\drp{u}{x_i}\drp{v}{x_j}+\sum_{i=1}^nb_i\drp{u}{x_i}v+cuv\biggr]\,\dd\mu
    \label{eq:forma-bilineare-iperboliche}
\end{equation}
definita da $\sobHc[1]{\Omega}\times\sobHc[1]{\Omega}$ a $\R$.
Diciamo che $u\in\leb[2]{0,T;\sobHc[1]{\Omega}}$, con $\drp{u}{t}\in\leb[2]{0,T;\leb[2]{\Omega}}$ e $\ddrp{u}{t}\in\leb[2]{0,T;\sobH[-1]{\Omega}}$ è soluzione debole del \eqref{eq:problema-cauchy-iperbolico} se per ogni $v\in\sobHc[1]{\Omega}$ vale
\begin{equation}
    \inner[\Big]{\ddrp{u}{t}}{v}_2+\inner{L(u)}{v}_2=\inner{f}{v}_2
\end{equation}
per quasi ogni $t\in[0,T]$, ossia
\begin{equation}
    \inner[\Big]{\ddrp{u}{t}}{v}_2+B(u,v,t)=\inner{f}{v}_2.
\end{equation}
e $u(0,\cdot)=g$, $\drp{u}{t}(0,\cdot)=h$.
Come per le equazioni paraboliche, possiamo anche intendere che $\ddrp{u}{t}\in\sobH[-1]{\Omega}$ sostituendo $\inner{\ddrp{u}{t}}{v}$ con la forma di dualità tra $\sobH[-1]{\Omega}$ e $\sobHc[1]{\Omega}$.
Anche il procedimento per dimostrare l'esistenza e l'unicità delle soluzioni deboli è analogo al caso parabolico.
Effettuiamo un'approssimazione di dimensione finita per poi ricondurci, con un limite, al caso originale: sia $\{e_k\}_{k\in\N}$ una base ortonormale in $\leb[2]{\Omega}$ e $\sobHc[1]{\Omega}$, e consideriamo
\begin{equation}
    u_m(t,\vec x)=\sum_{j=1}^md^j_m(t)e_k(\vec x)
    \label{eq:approssimazione-finita-iperboliche}
\end{equation}
per $d^j_m\colon[0,T]\to\R$ tali che $d^j_m(0)=\inner{g}{e_j}_2$ e $\drv{d^j_m}{t}(0)=\inner{h}{e_j}_2$ per ogni $j\in\{1,\dotsc,m\}$.
\begin{teorema} \label{t:esistenza-soluzione-approssimata-iperboliche}
    Per ogni $m\in\N$ esiste ed è unica una funzione $u_m$ della forma \eqref{eq:approssimazione-finita-iperboliche} che soddisfa
    \begin{equation}
        \dualpair[\Big]{\ddrp{u_m}{t}}{e_k}+B(u_m,e_k,t)=\inner{f}{e_k}_2
    \end{equation}
    per ogni $k\in\N$.
\end{teorema}
\begin{proof}
    Anche in questo caso ci riconduciamo a delle equazioni differenziali ordinarie: risulta che
    \begin{equation}
        \dualpair[\Big]{\ddrp{u}{t}}{e_k}=
        \sum_{j=1}^m\ddrv{d^j_m}{t}(t)\dualpair{e_j}{e_k}=
        \sum_{j=1}^m\ddrv{d^j_m}{t}(t)\inner{e_j}{e_k}_2=
        \ddrv{d^k_m}{t}(t)
    \end{equation}
    inoltre, posto $b_{jk}(t)=B(e_j,e_k,t)$,
    \begin{equation}
        B(u_m,e_k,t)=
        \sum_{j=1}^md^j_m(t)B(e_j,e_k,t)=
        \sum_{j=1}^md^j_m(t)b_{jk}(t)
    \end{equation}
    e infine scriviamo $f_k\defeq\inner{f}{e_k}_2$.
    Otteniamo in questo modo un sistema di equazioni differenziali lineari ordinarie del secondo ordine nella variabile $t$,
    \begin{equation}
        \ddrv{}{t}d^k_m(t)+\sum_{j=1}^mb_{jk}(t)d^j_m(t)=f_k(t)
    \end{equation}
    che con le opportune condizioni iniziali ammette un'unica soluzione, in $\cont[2]{(0,T)}$, per il teorema \ref{t:E-globale}.
\end{proof}
\begin{teorema} \label{t:stima-energia-iperboliche}
    Esiste una costante $c=c(\Omega,L,T)$ tale che
    \begin{multline}
        \max_{t\in[0,T]}\biggl(\norm{u_m(t)}_{\sobHc[1]{\Omega}}+\norm[\bigg]{\drp{u_m}{t}(t)}_{\leb[2]{\Omega}}\biggr)+\norm[\bigg]{\ddrp{u}{t}(t)}_{\leb[2]{0,T;\sobH[-1]{\Omega}}}\le\\ \le
        c\bigl(\norm{f}_{\leb[2]{0,T;\leb[2]{\Omega}}}+\norm{g}_{\sobHc[1]{\Omega}}+\norm{h}_{\leb[2]{\Omega}}\bigr).
        \label{eq:stima-energia-iperboliche}
    \end{multline}
\end{teorema}
\begin{teorema} \label{t:esistenza-unicita-iperboliche}
    Esiste ed è unica una soluzione debole del problema di Cauchy \eqref{eq:problema-cauchy-iperbolico}.
\end{teorema}
\begin{proof}
    Consideriamo la successione $\{u_m\}_{m\in\N}$ delle soluzioni approssimate \eqref{eq:approssimazione-finita-iperboliche}: dal teorema precedente
    \begin{equation}
        \norm{u_m}_{\leb[2]{0,T;\sobHc[1]{\Omega}}}=
        \int_0^T\int_\Omega\norm{\grad u_m}^2\,\dd\mu\,\dd t\le
        T\max_{t\in[0,T]}\int_\Omega\norm{\grad u_m(t)}^2\,\dd\mu\le
        Tc
    \end{equation}
    quindi per il teorema di Banach-Alao\u{g}lu \ref{t:banach-alaoglu} otteniamo che esiste una sottosuccessione $\{u_{m_k}\}$ tale che $u_{m_k}\weakto u\in\leb[2]{0,T;\sobHc[1]{\Omega}}$ per $m_k\to+\infty$.
    Analogamente $\drp{u_{m_k}}{t}\weakto\drp{u}{t}$ e $\ddrp{u_{m_k}}{t}\weakto\ddrp{u}{t}$ nei rispettivi spazi.
    Inoltre dall'equazione
    \begin{equation}
        \int_0^T\biggl[\dualpair[\Big]{\ddrp{u_{m_k}}{t}}{v}+B(u_{m_k},v,t)\biggr]\,\dd t=\int_0^T\inner{f}{v}_2\,\dd t,
    \end{equation}
    ponendo $v=\sum_{j=1}^mc_j(t)e_j$, per $m\to+\infty$ si ottiene la convergenza
    \begin{equation}
        \int_0^T\biggl[\dualpair[\Big]{\ddrp{u}{t}}{v}+B(u,v,t)\biggr]\,\dd t=\int_0^T\inner{f}{v}_2\,\dd t.
    \end{equation}
    e siccome l'insieme di tali $v$ approssimati è denso in $\sobHc[1]{\Omega}$ si ha anche che vale per ogni $v\in\sobHc[1]{\Omega}$.
    Di conseguenza vale che
    \begin{equation}
        \dualpair[\Big]{\ddrp{u}{t}}{v}+B(u,v,t)=\inner{f}{v}_2
    \end{equation}
    per quasi ogni $t\in[0,T]$, vale a dire $u$ è soluzione debole del problema di Cauchy.

    Per quanto riguarda l'unicità, se abbiamo due soluzioni del medesimo problema di Cauchy allora la loro differenza risolve l'equazione iperbolica ma con condizioni al contorno tutte nulle, ossia
    \begin{equation}
        \begin{cases}
            \dualpair[\big]{\ddrp{u}{t}}{v}+B(u,v,t)=0 & \text{in }\Omega_T=[0,T]\times\Omega\\
            u=0                                        & \text{in }[0,T]\times\boundary\Omega\cup\{0\}\times\Omega\\
            \drp{u}{t}=0                               & \text{in }\{0\}\times\Omega
        \end{cases}
    \end{equation}
    e mostriamo che la soluzione di tale problema è il solo zero, cos\`i che le due soluzioni di \eqref{eq:problema-cauchy-iperbolico} coincidano.
    Fissiamo $s\in[0,T]$ e definiamo la funzione
    \begin{equation}
        v(t)\defeq
        \begin{cases}
            \int_t^su(\tau)\,\dd\tau & t\in[0,s]\\
            0                        & t\in(s,T]
        \end{cases}
    \end{equation}
    che è in $\sobHc[1]{\Omega}$ per ogni $t\in[0,T]$; possiamo sostituirla nella forma debole dell'equazione differenziale ottenendo
    \begin{equation}
        \int_0^s\biggl[\dualpair[\Big]{\ddrp{u}{t}}{v}+B(u,v,t)\biggr]\,\dd t=0
    \end{equation}
    con $v(s)=0$ e $\drp{u}{t}(0)=0$.
    Integrando per parti, siccome i termini di bordo sono nulli, otteniamo
    \begin{equation}
        \int_0^s\biggl[-\dualpair[\Big]{\drp{u}{t}}{\drp{v}{t}}+B(u,v,t)\biggr]\,\dd t=0
    \end{equation}
    ma $\drp{v}{t}=-u$ se $0\le t\le s$ quindi
    \begin{equation}
        \int_0^s\biggl[\dualpair[\Big]{\drp{u}{t}}{u}+B\biggl(\drp{v}{t},v,t\biggr)\biggr]\,\dd t=0
    \end{equation}
    da cui ricaviamo
    \begin{equation}
        \int_0^s\biggl[\frac12\drv{}{t}\norm{u}_2^2-\frac12\drv{}{t}B\biggl(\drp{v}{t},v,t\biggr)\biggr]\,\dd t=
        -\int_0^s\bigl[C(u,v,t)+D(v,t)\bigr]\,\dd t
    \end{equation}
    dove $C$ e $D$ sono termini di resto provenienti dall'integrazione per parti della forma bilineare $B$:
    \begin{equation}
        \begin{gathered}
            C(u,v,t)=\int_\Omega\sum_{i=1}^n\biggl(b_iu\drp{v}{x_i}+\frac12\drp{b_i}{x_i}uv\biggr)\,\dd\mu\\
            D(v,t)=\frac12\int_\Omega\biggl(\sum_{i=1}^n\sum_{j=1}^n\drp{a_{ij}}{t}\drp{u}{x_i}\drp{v}{x_j}+\sum_{i=1}^n\drp{b_i}{t}\drp{u}{x_i}v+\drp{c}{t}uv\biggr)\,\dd\mu.
        \end{gathered}
    \end{equation}
    Infatti si ha che
    \begin{equation}
        \begin{split}
            \frac12\drp{}{t}B(v,v,t)&=
            \frac12\drp{}{t}\int_\Omega\biggl[\sum_{i=1}^n\sum_{j=1}^na_{ij}\drp{v}{x_i}\drp{v}{x_j}+\sum_{i=1}^nb_i\drp{v}{x_i}v+cv^2\biggr]\,\dd\mu=\\ &=
                \int_\Omega\sum_{i=1}^n\sum_{j=1}^na_{ij}\drp{}{t}\biggl(\drp{v}{x_i}\biggr)\drp{v}{x_j}\,\dd\mu+
                \frac12\int_\Omega\sum_{i=1}^nb_i\drp{}{t}\biggl(\drp{v}{x_i}\biggr)v\,\dd\mu+\\ &+
                \frac12\int_\Omega\sum_{i=1}^nb_i\drp{v}{x_i}\drp{v}{t}\,\dd\mu+
                \int_\Omega c\drp{v}{t}v\,\dd\mu+
                \frac12\int_\Omega\sum_{i=1}^n\sum_{j=1}^n\drp{a_{ij}}{t}\drp{v}{x_i}\drp{v}{x_j}\,\dd\mu+\\ &+
                \frac12\int_\Omega\sum_{i=1}^n\drp{b_i}{t}\drp{v}{x_i}v\,\dd\mu
                \frac12\int_\Omega\drp{c}{t}v^2\,\dd\mu
        \end{split}
        \label{eq:dim-esistenza-unicita-iperboliche1}
    \end{equation}
    ma siccome (tenendo a parte il fattore $\frac12$)
    \begin{equation}
        \int_\Omega\sum_{i=1}^nb_i\drp{v}{x_i}\drp{v}{t}\,\dd\mu=
        -\int_\Omega\sum_{i=1}^nb_i\drp{}{t}\biggl(\drp{v}{x_i}\biggr)v\,\dd\mu-\int_\Omega\sum_{i=1}^n\drp{b_i}{x_i}\drp{v}{x_i}v\,\dd\mu
    \end{equation}
    e il primo termine si cancella con uno dei precedenti dalla \eqref{eq:dim-esistenza-unicita-iperboliche1} ricaviamo che esistono $c,d>0$ tali per cui
    \begin{equation}
        B\bigl(v(0),v(0),t\bigr)=c\norm{v(0)}_{\sobHc[1]{\Omega}}^2+d\norm{v(0)}_2^2
    \end{equation}
    perciò
    \begin{equation}
        \norm{u(s)}_2^2+\norm{v(0)}_{\sobHc[1]{\Omega}}^2\le
        d\norm{v(0)}_2^2+\int_0^s\bigl[c\norm{v(t)}_{\sobHc[1]{\Omega}}^2+d\norm{v(t)}_2^2\bigr]\,\dd t.
    \end{equation}
    Poniamo dunque $w(t)\defeq\int_0^t u(s)\,\dd s$, per $t\in[0,T]$: dato che $v(0)=w(s)$, risulta
    \begin{equation}
        \norm{u(s)}_2^2+\norm{w(s)}_{\sobHc[1]{\Omega}}^2\le
        c\int_0^s\bigl[\norm{w(t)-w(s)}_{\sobHc[1]{\Omega}}^2+\norm{u(t)}_2^2\bigr]\,\dd t+\norm{w(s)}_2^2.
    \end{equation}
    Grazie alle disuguaglianze
    \begin{gather}
        \norm{w(t)-w(s)}_{\sobHc[1]{\Omega}}^2 \le 2\norm{w(t)}_{\sobHc[1]{\Omega}}^2+2\norm{w(s)}_{\sobHc[1]{\Omega}}^2\\
        \norm{w(s)}_2^2 \le \int_0^s\norm{u(t)}_2^2\,\dd t
    \end{gather}
    troviamo infine
    \begin{equation}
        \norm{u(s)}_2^2+(1-2cs)\norm{w(s)}_{\sobHc[1]{\Omega}}^2\le
        2c\int_0^s\bigl[\norm{w(t)}_{\sobHc[1]{\Omega}}^2+\norm{u(t)}_2^2\bigr]\,\dd t.
    \end{equation}
    Ora, preso $t\in[0,T']$ con $T'$ tale che $1-2cT'=\frac12$ troviamo che $1-2cs\ge\frac12$ e riaggiustando le costanti risulta
    \begin{equation}
        \norm{u(s)}_2^2+\norm{w(s)}_{\sobHc[1]{\Omega}}\le k\int_0^s\bigl[\norm{w(t)}_{\sobHc[1]{\Omega}}^2+\norm{u(t)}_2^2\bigr]\,\dd t
    \end{equation}
    per un certo $k>0$, da cui $u=w=0$ su $[0,T']$.
    Iterando questo procedimento su ogni intervallo $[T',T'']$, poi $[T'',T''']$ e cos\`i via fino a ricoprire tutto $[0,T]$ troviamo che la soluzione $u$ è nulla su tutto tale intervallo, provando cos\`i che la soluzione debole è unica.
\end{proof}
