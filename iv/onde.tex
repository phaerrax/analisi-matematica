\chapter{L'equazione delle onde}
\label{ch:onde}

L'equazione delle onde, o anche di D'Alembert, è
\begin{equation}
    \ddrp{u}{t}-\lap u=f
    \label{eq:onde}
\end{equation}
per $u=u(t,\vec x)$ con $t\in[0,+\infty)$ e $\vec x\in\Omega\subset\R^n$.
Più in breve, definendo l'\emph{operatore dalembertiano} $\dal=\ddrp{}{t}-\lap$ possiamo riscriverla come
\begin{equation}
    \dal u=f.
    \label{eq:onde-dalembertiano}
\end{equation}
Per semplicità, trattiamo il caso $n=1$, $\Omega=\R$ e poniamo il problema di Cauchy
\begin{equation}
    \begin{cases}
        \ddrp{u}{t}-\ddrp{u}{x}=0 & \text{in }(0,+\infty)\times\R\\
        u=g                       & \text{in }\{0\}\times\R\\
        \drp{u}{t}=h              & \text{in }\{0\}\times\R
    \end{cases}
    \label{eq:problema-cauchy-onde}
\end{equation}
e cerchiamo una soluzione in termini di $g$ e $h$.
Mediante un'opportuna fattorizzazione possiamo ricondurre l'equazione omogenea a una coppia di equazioni del trasporto:
\begin{equation}
    \dal u=\biggl(\drp{}{t}+\drp{}{x}\biggr)\biggl(\drp{u}{t}-\drp{u}{x}\biggr)=0
\end{equation}
e detto $v=\drp{u}{t}-\drp{u}{x}$ abbiamo
\begin{equation}
    \drp{v}{t}+\drp{v}{x}=0.
\end{equation}
Riconducendoci all'esempio \ref{es:equazione-trasporto}, sia $a(x-t)\defeq v(t,x)$; risulta $\drp{v}{x}(t,x)=a'(x-t)$ e $\drp{v}{t}(t,x)=-a'(t,x)$.
D'altro canto l'equazione
\begin{equation}
    \drp{u}{t}(t,x)-\drp{u}{x}(t,x)=a(x-t)
\end{equation}
è ancora un'equazione del trasporto (non omogenea), per cui
\begin{equation}
    u(t,x)=
    \int_0^ta\bigl(x+(t-s)-s)\,\dd s+b(x+t)=
    \frac12\int_{x-t}^{x+t}a(y)\,\dd y+b(x+t)
\end{equation}
dove $a$ è determinata dalla non omogeneità della soluzione mentre $b$ è determinata dai dati iniziali: si ha $b(x)=u(0,x)=g(x)$ e $a(x)=v(0,x)=\drp{u}{t}(0,x)-\drp{u}{x}(0,x)=h(x)-g'(x)$ quindi
\begin{equation}
    \begin{split}
        u(t,x)&=
        \frac12\int_{x-t}^{x+t}\bigl[h(y)-g'(y)\bigr]\,\dd y+g(x+t)=\\ &=
        \frac12\int_{x-t}^{x+t}h(y)\,\dd y+\frac12\bigl[g(x-t)-g(x+t)\bigr]+g(x+t)=\\ &=
        \frac12\int_{x-t}^{x+t}h(y)\,\dd y+\frac12\bigl[g(x-t)+g(x+t)\bigr]
    \end{split}
    \label{eq:rappresentazione-soluzione-onde}
\end{equation}
che è la formula di rappresentazione generale.
Essa vale per $t>0$, ma si dimostra che se $g\in\cont[2]{\R}$ e $h\in\cont[1]{\R}$ allora $u$ soddisfa l'equazione delle onde su tutto $\R$.
\begin{teorema}
    Se $g\in\cclass[2]$ e $h\in\cclass[1]$ allora la soluzione $u$ del problema di Cauchy \eqref{eq:problema-cauchy-onde} è in $\cont[2]{(0,+\infty)\times\R}$ e vale
    \begin{equation}
        g(x_0)=\lim_{(t,x)\to(0,x_0)}u(t,x)
        \qtext{e}
        h(x_0)=\lim_{(t,x)\to(0,x_0)}\drp{u}{t}(t,x).
    \end{equation}
\end{teorema}

\section{Equazioni iperboliche}
Generalizziamo l'equazione delle onde a un generico problema
\begin{equation}
    \begin{cases}
        \ddrp{u}{t}+L(u)=f & \text{in }\Omega_T=[0,T]\times\Omega\\
        u=0                & \text{in }[0,T]\times\boundary\Omega\\
        u=g                & \text{in }\{0\}\times\Omega\\
        \drp{u}{t}=h       & \text{in }\{0\}\times\Omega
    \end{cases}
    \label{eq:problema-cauchy-iperbolico}
\end{equation}
con $f\colon\Omega_T\to\R$ e $g,h\colon\Omega\to\R$, ed $L$ un operatore ellittico dipendente da $t$:
\begin{equation}
    L\colon u\mapsto -\sum_{i=1}^n\sum_{j=1}^n\drp{}{x_i}\biggl(a_{ij}\drp{u}{x_j}\biggr)+\sum_{i=1}^nb_i\drp{u}{x_i}+cu
\end{equation}
tale che esiste $\lambda>0$ per cui
\begin{equation}
    \sum_{i=1}^n\sum_{j=1}^n a_{ij}(t,\vec x)\xi_i\xi_j \ge \lambda\norm{\vxi}^2
\end{equation}
per ogni $\vxi\in\R^n$ e $(t,\vec x)\in\Omega_T$.
Cerchiamone una soluzione debole, assumendo che $a_{ji}=a_{ij}$ e $a_{ij},b_i\in\cont[1]{\clos{\Omega_T}}$, $f\in\leb[2]{\Omega_T}$, $g\in\sobHc[1]{\Omega}$ e $h\in\leb[2]{\Omega}$.
Ricaviamo dall'equazione la forma bilineare
\begin{equation}
    B(u,v,t)=\int_\Omega\biggl[\sum_{i=1}^n\sum_{j=1}^na_{ij}\drp{u}{x_i}\drp{v}{x_j}+\sum_{i=1}^nb_i\drp{u}{x_i}v+cuv\biggr]\,\dd\mu
    \label{eq:forma-bilineare-iperboliche}
\end{equation}
definita da $\sobHc[1]{\Omega}\times\sobHc[1]{\Omega}$ a $\R$.
Diciamo che $u\in\leb[2]{0,T;\sobHc[1]{\Omega}}$, con $\drp{u}{t}\in\leb[2]{0,T;\leb[2]{\Omega}}$ e $\ddrp{u}{t}\in\leb[2]{0,T;\sobH[-1]{\Omega}}$ è soluzione debole del \eqref{eq:problema-cauchy-iperbolico} se per ogni $v\in\sobHc[1]{\Omega}$ vale
\begin{equation}
    \inner[\Big]{\ddrp{u}{t}}{v}_2+\inner{L(u)}{v}_2=\inner{f}{v}_2
\end{equation}
per quasi ogni $t\in[0,T]$, ossia
\begin{equation}
    \inner[\Big]{\ddrp{u}{t}}{v}_2+B(u,v,t)=\inner{f}{v}_2.
\end{equation}
e $u(0,\cdot)=g$, $\drp{u}{t}(0,\cdot)=h$.
Come per le equazioni paraboliche, possiamo anche intendere che $\ddrp{u}{t}\in\sobH[-1]{\Omega}$ sostituendo $\inner{\ddrp{u}{t}}{v}$ con la forma di dualità tra $\sobH[-1]{\Omega}$ e $\sobHc[1]{\Omega}$.
Anche il procedimento per dimostrare l'esistenza e l'unicità delle soluzioni deboli è analogo al caso parabolico.
Effettuiamo un'approssimazione di dimensione finita per poi ricondurci, con un limite, al caso originale: sia $\{e_k\}_{k\in\N}$ una base ortonormale in $\leb[2]{\Omega}$ e $\sobHc[1]{\Omega}$, e consideriamo
\begin{equation}
    u_m(t,\vec x)=\sum_{j=1}^md^j_m(t)e_k(\vec x)
    \label{eq:approssimazione-finita-iperboliche}
\end{equation}
per $d^j_m\colon[0,T]\to\R$ tali che $d^j_m(0)=\inner{g}{e_j}_2$ e $\drv{d^j_m}{t}(0)=\inner{h}{e_j}_2$ per ogni $j\in\{1,\dotsc,m\}$.
\begin{teorema} \label{t:esistenza-soluzione-approssimata-iperboliche}
    Per ogni $m\in\N$ esiste ed è unica una funzione $u_m$ della forma \eqref{eq:approssimazione-finita-iperboliche} che soddisfa
    \begin{equation}
        \dualpair[\Big]{\ddrp{u_m}{t}}{e_k}+B(u_m,e_k,t)=\inner{f}{e_k}_2
    \end{equation}
    per ogni $k\in\N$.
\end{teorema}
\begin{proof}
    Anche in questo caso ci riconduciamo a delle equazioni differenziali ordinarie: risulta che
    \begin{equation}
        \dualpair[\Big]{\ddrp{u}{t}}{e_k}=
        \sum_{j=1}^m\ddrv{d^j_m}{t}(t)\dualpair{e_j}{e_k}=
        \sum_{j=1}^m\ddrv{d^j_m}{t}(t)\inner{e_j}{e_k}_2=
        \ddrv{d^k_m}{t}(t)
    \end{equation}
    inoltre, posto $b_{jk}(t)=B(e_j,e_k,t)$,
    \begin{equation}
        B(u_m,e_k,t)=
        \sum_{j=1}^md^j_m(t)B(e_j,e_k,t)=
        \sum_{j=1}^md^j_m(t)b_{jk}(t)
    \end{equation}
    e infine scriviamo $f_k\defeq\inner{f}{e_k}_2$.
    Otteniamo in questo modo un sistema di equazioni differenziali lineari ordinarie del secondo ordine nella variabile $t$,
    \begin{equation}
        \ddrv{}{t}d^k_m(t)+\sum_{j=1}^mb_{jk}(t)d^j_m(t)=f_k(t)
    \end{equation}
    che con le opportune condizioni iniziali ammette un'unica soluzione, in $\cont[2]{(0,T)}$, per il teorema \ref{t:E-globale}.
\end{proof}
\begin{teorema} \label{t:stima-energia-iperboliche}
    Esiste una costante $c=c(\Omega,L,T)$ tale che
    \begin{multline}
        \max_{t\in[0,T]}\biggl(\norm{u_m(t)}_{\sobHc[1]{\Omega}}+\norm[\bigg]{\drp{u_m}{t}(t)}_{\leb[2]{\Omega}}\biggr)+\norm[\bigg]{\ddrp{u}{t}(t)}_{\leb[2]{0,T;\sobH[-1]{\Omega}}}\le\\ \le
        c\bigl(\norm{f}_{\leb[2]{0,T;\leb[2]{\Omega}}}+\norm{g}_{\sobHc[1]{\Omega}}+\norm{h}_{\leb[2]{\Omega}}\bigr).
        \label{eq:stima-energia-iperboliche}
    \end{multline}
\end{teorema}
